\documentclass[12pt]{beamer}\usepackage[]{graphicx}\usepackage[]{color}
%% maxwidth is the original width if it is less than linewidth
%% otherwise use linewidth (to make sure the graphics do not exceed the margin)
\makeatletter
\def\maxwidth{ %
  \ifdim\Gin@nat@width>\linewidth
    \linewidth
  \else
    \Gin@nat@width
  \fi
}
\makeatother

\definecolor{fgcolor}{rgb}{0.345, 0.345, 0.345}
\newcommand{\hlnum}[1]{\textcolor[rgb]{0.686,0.059,0.569}{#1}}%
\newcommand{\hlstr}[1]{\textcolor[rgb]{0.192,0.494,0.8}{#1}}%
\newcommand{\hlcom}[1]{\textcolor[rgb]{0.678,0.584,0.686}{\textit{#1}}}%
\newcommand{\hlopt}[1]{\textcolor[rgb]{0,0,0}{#1}}%
\newcommand{\hlstd}[1]{\textcolor[rgb]{0.345,0.345,0.345}{#1}}%
\newcommand{\hlkwa}[1]{\textcolor[rgb]{0.161,0.373,0.58}{\textbf{#1}}}%
\newcommand{\hlkwb}[1]{\textcolor[rgb]{0.69,0.353,0.396}{#1}}%
\newcommand{\hlkwc}[1]{\textcolor[rgb]{0.333,0.667,0.333}{#1}}%
\newcommand{\hlkwd}[1]{\textcolor[rgb]{0.737,0.353,0.396}{\textbf{#1}}}%
\let\hlipl\hlkwb

\usepackage{framed}
\makeatletter
\newenvironment{kframe}{%
 \def\at@end@of@kframe{}%
 \ifinner\ifhmode%
  \def\at@end@of@kframe{\end{minipage}}%
  \begin{minipage}{\columnwidth}%
 \fi\fi%
 \def\FrameCommand##1{\hskip\@totalleftmargin \hskip-\fboxsep
 \colorbox{shadecolor}{##1}\hskip-\fboxsep
     % There is no \\@totalrightmargin, so:
     \hskip-\linewidth \hskip-\@totalleftmargin \hskip\columnwidth}%
 \MakeFramed {\advance\hsize-\width
   \@totalleftmargin\z@ \linewidth\hsize
   \@setminipage}}%
 {\par\unskip\endMakeFramed%
 \at@end@of@kframe}
\makeatother

\definecolor{shadecolor}{rgb}{.97, .97, .97}
\definecolor{messagecolor}{rgb}{0, 0, 0}
\definecolor{warningcolor}{rgb}{1, 0, 1}
\definecolor{errorcolor}{rgb}{1, 0, 0}
\newenvironment{knitrout}{}{} % an empty environment to be redefined in TeX

\usepackage{alltt}
\usepackage{tikz}

% make it pretty
% get rid of junk
\usetheme{default}
\usefonttheme[onlymath]{serif}
\beamertemplatenavigationsymbolsempty

% define a bunch of colors
\definecolor{offwhite}{RGB}{255,250,240}
\definecolor{gray}{RGB}{155,155,155}
\definecolor{foreground}{RGB}{80,80,80}
\definecolor{background}{RGB}{255,255,255}
%\definecolor{title}{RGB}{255,199,0}
\definecolor{title}{RGB}{89,132,212}
%\definecolor{subtitle}{RGB}{89,132,212}
\definecolor{subtitle}{RGB}{255,199,0}
\definecolor{hilit}{RGB}{248,117,79}
\definecolor{vhilit}{RGB}{255,111,207}
\definecolor{lolit}{RGB}{200,200,200}
\definecolor{lit}{RGB}{255,199,0}
\definecolor{mdlit}{RGB}{89,132,212}
\definecolor{link}{RGB}{248,117,79}

% a few color macros
\newcommand{\hilit}{\color{hilit}}
\newcommand{\vhilit}{\color{vhilit}}
\newcommand{\lit}{\color{lit}}
\newcommand{\mdlit}{\color{mdlit}}
\newcommand{\lolit}{\color{lolit}}

% use those colors
\setbeamercolor{titlelike}{fg=title}
\setbeamercolor{subtitle}{fg=subtitle}
\setbeamercolor{frametitle}{fg=gray}
%\setbeamercolor{structure}{fg=subtitle}
\setbeamercolor{structure}{fg=title}
\setbeamercolor{institute}{fg=lolit}
\setbeamercolor{normal text}{fg=foreground,bg=background}
\setbeamertemplate{itemize subitem}{{\textendash}}
\setbeamerfont{itemize/enumerate subbody}{size=\small}
\setbeamerfont{itemize/enumerate subitem}{size=\small}

% center title of slides
\setbeamertemplate{blocks}[rounded]
\setbeamertemplate{frametitle}[default][center]

% page number
\setbeamerfont{page number in foot}{size=\footnotesize}
\setbeamertemplate{footline}[frame number]

% default link color
\hypersetup{colorlinks, urlcolor={link}}

% a few macros
\newcommand{\code}[1]{\texttt{#1}}
\newcommand{\hicode}[1]{{\hilit \texttt{#1}}}
\newcommand{\locode}[1]{{\lolit \texttt{#1}}}
\newcommand{\bb}[1]{\begin{block}{#1}}
\newcommand{\eb}{\end{block}}
\newcommand{\bi}{\begin{itemize}}
\newcommand{\bbi}{\vspace{4pt} \begin{itemize} \itemsep8pt}
\newcommand{\ei}{\end{itemize}}
\newcommand{\bv}{\begin{verbatim}}
\newcommand{\ev}{\end{verbatim}}
\newcommand{\ig}{\includegraphics}
\newcommand{\subt}[1]{{\footnotesize \color{subtitle} {#1}}}
\newcommand{\ttsm}{\tt \small}
\newcommand{\ttfn}{\tt \footnotesize}
\newcommand{\figh}[2]{\centerline{\includegraphics[height=#2\textheight]{#1}}}
\newcommand{\figw}[2]{\centerline{\includegraphics[width=#2\textwidth]{#1}}}



%------------------------------------------------

\title{Matrices in R}
\subtitle{Matrix Algebra 4 Statistical Learning}
\author{\href{http://www.gastonsanchez.com}{Gaston Sanchez}}
\institute{\href{https://creativecommons.org/licenses/by-sa/4.0/}{\tt \scriptsize \color{foreground} CC BY-SA 4.0}}
\date{}
\IfFileExists{upquote.sty}{\usepackage{upquote}}{}
\begin{document}



% no page number in first slide
{
  \setbeamertemplate{footline}{} 
  \frame{\titlepage} 
}

%------------------------------------------------

\begin{frame}
\begin{center}
\Huge{\hilit{Why Matrix Algebra?}}
\end{center}
\end{frame}

%------------------------------------------------

\begin{frame}
\frametitle{Data Matrix}

\begin{block}{Data}
The analyzed data can be expressed in matrix format $\mathbf{X}$:

\[ \underset{n \times p}{\mathbf{X}} = 
\left[\begin{array}{cccc}
x_{11} & x_{12} & \cdots & x_{1p} \\
x_{21} & x_{22} & \cdots & x_{2p} \\
\vdots & \vdots & \ddots & \vdots \\
x_{n1} & x_{n2} & \cdots & x_{np} \\
\end{array}\right]
\]
\end{block}

\begin{itemize}
 \item $n$ objects in the rows
 \item $p$ variables in the columns
\end{itemize}

\end{frame}

%------------------------------------------------

\begin{frame}
\frametitle{Motivation}

\bbi
  \item Multivariate data is commonly represented in tabular format (rows and columns)
  \item Mathematically, a data table can be treated as a matrix
  \item Matrix algebra provides the analytical machinery and tools to
manipulate and exploit values, information, and patterns of variability in data
\ei

{\lit In these slides I will cover key concepts about matrix operations with R.}

\end{frame}

%------------------------------------------------

\begin{frame}
\frametitle{Vector-Matrix Notation}

\bi
  \item upper-case italics for variables: $X, Y$
  \item bold upper-case for matrices: $\mathbf{A}, \mathbf{B}$
  \item bold lower-case letters for vectors: $\mathbf{x}, \mathbf{y}$
  \item lower-case italics are used to represent scalars: $a, b$
  \item $x_i$ is the $i$-th element of $\mathbf{x}$
  \item $A_{ij}$ is the element in the $i$-th row and $j$-th column of $\mathbf{A}$
  \item By default, we'll consider a vector $\mathbf{x}$ to be a one-column matrix
  \item $\mathbf{x}^\mathsf{T}$ to be a one-row matrix
  \item Occasionally I'll be using some greek letters too
\ei

\end{frame}

%------------------------------------------------

\begin{frame}
\frametitle{Two Assumptions}

\centerline{\mdlit \Large I'm assuming 2 things about you:}

\bigskip
\centerline{\Large Matrix Algebra \quad \& \quad R basics}

\end{frame}

%------------------------------------------------

\begin{frame}
\frametitle{Matrix Algebra}

You should have been exposed to concepts such as:

\bi
  \item Vector Spaces
  \item Inner Products
  \item Matrix Multiplication
  \item Linear Dependency
  \item Rank
  \item Trace, Determinant
  \item Inverse
  \item \textit{etc}
\ei

\end{frame}

%------------------------------------------------

\begin{frame}
\frametitle{R Basics}

You should have been exposed to:

\bi
  \item R \code{vector}'s, \code{list}'s, \code{data.frame}'s
  \item Subscripting and indexing (i.e. bracket notation)
  \item Writing functions: \code{function() \{...\}}
  \item Conditionals: \code{if \{...\} else \{...\}}
  \item Loops: \code{for}, \code{while}, \code{repeat}
  \item Graphics: base, ggplot2, etc
  \item RStudio familiarity
\ei

\end{frame}

%------------------------------------------------

\begin{frame}[fragile]
\frametitle{Same but different}

\begin{columns}[t]
\begin{column}{0.1\textwidth}
%--- empty space ---%
\end{column}

\begin{column}{0.4\textwidth}
{\Large
{\mdlit Math Objects} \\
\bi
  \item vector
  \item matrix
\ei
}
\end{column}

\pause

\begin{column}{0.4\textwidth}
{\Large
{\mdlit R objects} \\
\bi
  \item \code{vector}
  \item \code{matrix}
\ei
}
\end{column}

\begin{column}{0.1\textwidth}
%--- empty space ---%
\end{column}
\end{columns}

\end{frame}

%------------------------------------------------

\begin{frame}
\begin{center}
\Huge{\hilit{Back to the (R) Basics}}
\end{center}
\end{frame}

%------------------------------------------------

\begin{frame}
\frametitle{R Vectors}

\bbi
  \item A vector is the most basic data structure in R
  \item Vectors are contiguous cells containing data
  \item Can be of any length (including zero)
  \item There are really no scalars, just one-element vectors
\ei

\end{frame}

%------------------------------------------------

\begin{frame}[fragile]
\frametitle{Vectors}

The most simple type of vectors are ``scalars'' or single values:
\begin{knitrout}\footnotesize
\definecolor{shadecolor}{rgb}{0.969, 0.969, 0.969}\color{fgcolor}\begin{kframe}
\begin{alltt}
\hlcom{# integer}
\hlstd{x} \hlkwb{<-} \hlnum{1L}
\hlcom{# double (real)}
\hlstd{y} \hlkwb{<-} \hlnum{5}
\hlcom{# complex}
\hlstd{z} \hlkwb{<-} \hlnum{3} \hlopt{+} \hlnum{5i}
\hlcom{# logical}
\hlstd{a} \hlkwb{<-} \hlnum{TRUE}
\hlcom{# character}
\hlstd{b} \hlkwb{<-} \hlstr{"yosemite"}
\end{alltt}
\end{kframe}
\end{knitrout}

{\lolit although keep in mind that R does NOT really have scalars}

\end{frame}

%------------------------------------------------

\begin{frame}
\frametitle{Data modes (i.e. types)}

\bbi
  \item A \textbf{double} vector stores regular (i.e. real) numbers
  \item An \textbf{integer} vector stores integers (no decimal component)
  \item A \textbf{character} vector stores text
  \item A \textbf{logical} vector stores \code{TRUE}'s and \code{FALSE}'s values
  \item A \textbf{complex} vector stores complex numbers
\ei

\end{frame}

%------------------------------------------------

\begin{frame}[fragile]
\frametitle{Special Values}

There are some special values 
\bbi
  \item \code{NULL} is the null object (it has length zero)
  \item Missing values are referred to by the symbol \code{NA} \\
  {\lolit (there are different modes of \code{NA}: logical, integer, etc)}
  \item \code{Inf} indicates positive infinite
  \item \code{-Inf} indicates negative infinite
  \item \code{NaN} indicates Not a Number \\
  {\lolit (don't confuse \code{NaN} with \code{NA})}

\ei

\end{frame}

%------------------------------------------------

\begin{frame}[fragile]
\frametitle{Vectors}

The function to create a vector from individual values is {\hilit \code{c()}}, 
short for \textit{concatenate} or \textit{combine}:

\begin{knitrout}\footnotesize
\definecolor{shadecolor}{rgb}{0.969, 0.969, 0.969}\color{fgcolor}\begin{kframe}
\begin{alltt}
\hlcom{# some vectors}
\hlstd{x} \hlkwb{<-} \hlkwd{c}\hlstd{(}\hlnum{1}\hlstd{,} \hlnum{2}\hlstd{,} \hlnum{3}\hlstd{,} \hlnum{4}\hlstd{,} \hlnum{5}\hlstd{)}

\hlstd{y} \hlkwb{<-} \hlkwd{c}\hlstd{(}\hlstr{"one"}\hlstd{,} \hlstr{"two"}\hlstd{,} \hlstr{"three"}\hlstd{)}

\hlstd{z} \hlkwb{<-} \hlkwd{c}\hlstd{(}\hlnum{TRUE}\hlstd{,} \hlnum{FALSE}\hlstd{,} \hlnum{FALSE}\hlstd{)}
\end{alltt}
\end{kframe}
\end{knitrout}
Separate each element by a comma

\end{frame}

%------------------------------------------------

\begin{frame}
\frametitle{Atomic Vectors}

\bbi
  \item vectors are \textbf{atomic} structures
  \item the values in a vector must be \textbf{ALL} of the same type 
  \item either all integers, or reals, or complex, or characters, or logicals
  \item you cannot have a vector of different data types
\ei

\end{frame}

%------------------------------------------------

\begin{frame}[fragile]
\frametitle{Atomic Vectors}

If you mix different data values, R will \textbf{implicitly coerce} them 
so they are all of the same type
\begin{knitrout}\footnotesize
\definecolor{shadecolor}{rgb}{0.969, 0.969, 0.969}\color{fgcolor}\begin{kframe}
\begin{alltt}
\hlcom{# mixing numbers and characters}
\hlstd{x} \hlkwb{<-} \hlkwd{c}\hlstd{(}\hlnum{1}\hlstd{,} \hlnum{2}\hlstd{,} \hlnum{3}\hlstd{,} \hlstr{"four"}\hlstd{,} \hlstr{"five"}\hlstd{)}
\hlstd{x}
\end{alltt}
\begin{verbatim}
## [1] "1"    "2"    "3"    "four" "five"
\end{verbatim}
\begin{alltt}
\hlcom{# mixing numbers and logical values}
\hlstd{y} \hlkwb{<-} \hlkwd{c}\hlstd{(}\hlnum{TRUE}\hlstd{,} \hlnum{FALSE}\hlstd{,} \hlnum{3}\hlstd{,} \hlnum{4}\hlstd{)}
\hlstd{y}
\end{alltt}
\begin{verbatim}
## [1] 1 0 3 4
\end{verbatim}
\end{kframe}
\end{knitrout}

\end{frame}

%------------------------------------------------

\begin{frame}
\frametitle{How does R coerce data types?}

\bb{R follows two basic rules of implicit coercion}
If a character is present, R will coerce everything else to characters

\pause
\bigskip

If a vector contains logicals and numbers, R will convert the logicals to 
numbers (\code{TRUE} to 1, \code{FALSE} to 0)
\eb

\end{frame}

%------------------------------------------------

\begin{frame}
\frametitle{Properties of Vectors}

\bbi
  \item all vectors have a length
  \item vector elements can have associated names
  \item vectors are objects of class \code{"vector"}
  \item vectors have a mode (storage mode)
\ei

\end{frame}

%------------------------------------------------

\begin{frame}[fragile]
\frametitle{Vectorization}

A vectorized computation is any computation that when applied to a 
vector operates on all of its elements

\begin{knitrout}\footnotesize
\definecolor{shadecolor}{rgb}{0.969, 0.969, 0.969}\color{fgcolor}\begin{kframe}
\begin{alltt}
\hlkwd{c}\hlstd{(}\hlnum{1}\hlstd{,} \hlnum{2}\hlstd{,} \hlnum{3}\hlstd{)} \hlopt{+} \hlkwd{c}\hlstd{(}\hlnum{3}\hlstd{,} \hlnum{2}\hlstd{,} \hlnum{1}\hlstd{)}
\end{alltt}
\begin{verbatim}
## [1] 4 4 4
\end{verbatim}
\begin{alltt}
\hlkwd{c}\hlstd{(}\hlnum{1}\hlstd{,} \hlnum{2}\hlstd{,} \hlnum{3}\hlstd{)} \hlopt{*} \hlkwd{c}\hlstd{(}\hlnum{3}\hlstd{,} \hlnum{2}\hlstd{,} \hlnum{1}\hlstd{)}
\end{alltt}
\begin{verbatim}
## [1] 3 4 3
\end{verbatim}
\end{kframe}
\end{knitrout}

\end{frame}

%------------------------------------------------

\begin{frame}[fragile]
\frametitle{Recycling}

When vectorized computations are applied, some problems may occur 
when dealing with two vectors of different length

\begin{knitrout}\footnotesize
\definecolor{shadecolor}{rgb}{0.969, 0.969, 0.969}\color{fgcolor}\begin{kframe}
\begin{alltt}
\hlkwd{c}\hlstd{(}\hlnum{1}\hlstd{,} \hlnum{2}\hlstd{,} \hlnum{3}\hlstd{,} \hlnum{4}\hlstd{)} \hlopt{+} \hlkwd{c}\hlstd{(}\hlnum{1}\hlstd{,} \hlnum{2}\hlstd{)}
\end{alltt}
\begin{verbatim}
## [1] 2 4 4 6
\end{verbatim}
\begin{alltt}
\hlkwd{c}\hlstd{(}\hlnum{2}\hlstd{,} \hlnum{1}\hlstd{)} \hlopt{+} \hlkwd{c}\hlstd{(}\hlnum{1}\hlstd{,} \hlnum{2}\hlstd{,} \hlnum{3}\hlstd{)}
\end{alltt}


{\ttfamily\noindent\color{warningcolor}{\#\# Warning in c(2, 1) + c(1, 2, 3): longer object length is not a multiple of shorter object length}}\begin{verbatim}
## [1] 3 3 5
\end{verbatim}
\end{kframe}
\end{knitrout}

\end{frame}

%------------------------------------------------

\begin{frame}[fragile]
\frametitle{Recycling}

The recycling rule can be very useful, like when operating between a vector and a ``scalar''

\begin{knitrout}\footnotesize
\definecolor{shadecolor}{rgb}{0.969, 0.969, 0.969}\color{fgcolor}\begin{kframe}
\begin{alltt}
\hlstd{x} \hlkwb{<-} \hlkwd{c}\hlstd{(}\hlnum{2}\hlstd{,} \hlnum{4}\hlstd{,} \hlnum{6}\hlstd{,} \hlnum{8}\hlstd{)}
\hlstd{x} \hlopt{+} \hlnum{3}
\end{alltt}
\begin{verbatim}
## [1]  5  7  9 11
\end{verbatim}
\end{kframe}
\end{knitrout}

\end{frame}

%------------------------------------------------

\begin{frame}
\begin{center}
\Huge{\hilit{R Matrices and Arrays}}
\end{center}
\end{frame}

%------------------------------------------------

\begin{frame}[fragile]
\frametitle{From Vectors to Arrays}

We can transform a vector in an \textbf{n-dimensional} array by giving it a dimensions attribute \code{dim}
\begin{knitrout}\footnotesize
\definecolor{shadecolor}{rgb}{0.969, 0.969, 0.969}\color{fgcolor}\begin{kframe}
\begin{alltt}
\hlcom{# positive: from 1 to 8}
\hlstd{x} \hlkwb{<-} \hlnum{1}\hlopt{:}\hlnum{8}

\hlcom{# adding 'dim' attribute}
\hlkwd{dim}\hlstd{(x)} \hlkwb{<-} \hlkwd{c}\hlstd{(}\hlnum{2}\hlstd{,} \hlnum{4}\hlstd{)}
\hlstd{x}
\end{alltt}
\begin{verbatim}
##      [,1] [,2] [,3] [,4]
## [1,]    1    3    5    7
## [2,]    2    4    6    8
\end{verbatim}
\end{kframe}
\end{knitrout}

\end{frame}

%------------------------------------------------

\begin{frame}[fragile]
\frametitle{From Vectors to Arrays}

\bbi
  \item a vector can be given a \code{dim} attribute (dimensions)
  \item a \code{dim} attribute is a numeric vector of length \code{n}
  \item R will reorganize the elements of the vector into \code{n} dimensions
  \item each dimension will have as many rows (or columns, etc.) as the \code{n}-th value of the \code{dim} vector
\ei

\end{frame}

%------------------------------------------------

\begin{frame}[fragile]
\frametitle{From Vectors to Arrays}

\begin{knitrout}\footnotesize
\definecolor{shadecolor}{rgb}{0.969, 0.969, 0.969}\color{fgcolor}\begin{kframe}
\begin{alltt}
\hlcom{# dim attribute with 3 dimensions}
\hlkwd{dim}\hlstd{(x)} \hlkwb{<-} \hlkwd{c}\hlstd{(}\hlnum{2}\hlstd{,} \hlnum{2}\hlstd{,} \hlnum{2}\hlstd{)}
\hlstd{x}
\end{alltt}
\begin{verbatim}
## , , 1
## 
##      [,1] [,2]
## [1,]    1    3
## [2,]    2    4
## 
## , , 2
## 
##      [,1] [,2]
## [1,]    5    7
## [2,]    6    8
\end{verbatim}
\end{kframe}
\end{knitrout}

\end{frame}

%------------------------------------------------

\begin{frame}[fragile]
\frametitle{From Vector to Matrix}

A \code{dim} attribute of length 2 will convert a vector into a matrix
\begin{knitrout}\footnotesize
\definecolor{shadecolor}{rgb}{0.969, 0.969, 0.969}\color{fgcolor}\begin{kframe}
\begin{alltt}
\hlcom{# vector to matrix}
\hlstd{A} \hlkwb{<-} \hlnum{1}\hlopt{:}\hlnum{8}
\hlkwd{class}\hlstd{(A)}
\end{alltt}
\begin{verbatim}
## [1] "integer"
\end{verbatim}
\begin{alltt}
\hlkwd{dim}\hlstd{(A)} \hlkwb{<-} \hlkwd{c}\hlstd{(}\hlnum{2}\hlstd{,} \hlnum{4}\hlstd{)}
\hlkwd{class}\hlstd{(A)}
\end{alltt}
\begin{verbatim}
## [1] "matrix"
\end{verbatim}
\end{kframe}
\end{knitrout}

When using \code{dim()}, R always fills up each matrix by columns.

\end{frame}

%------------------------------------------------

\begin{frame}[fragile]
\frametitle{From Vector to Matrix}

To have more control about how a matrix is filled,
we use the \code{matrix()} function:
\begin{knitrout}\footnotesize
\definecolor{shadecolor}{rgb}{0.969, 0.969, 0.969}\color{fgcolor}\begin{kframe}
\begin{alltt}
\hlcom{# vector to matrix (by rows)}
\hlstd{a} \hlkwb{<-} \hlnum{1}\hlopt{:}\hlnum{8}

\hlstd{A} \hlkwb{<-} \hlkwd{matrix}\hlstd{(a,} \hlkwc{nrow} \hlstd{=} \hlnum{2}\hlstd{,} \hlkwc{ncol} \hlstd{=} \hlnum{4}\hlstd{)}
\hlstd{A}
\end{alltt}
\begin{verbatim}
##      [,1] [,2] [,3] [,4]
## [1,]    1    3    5    7
## [2,]    2    4    6    8
\end{verbatim}
\end{kframe}
\end{knitrout}

\end{frame}

%------------------------------------------------

\begin{frame}[fragile]
\frametitle{From Vector to Matrix}

If you want to fill a matrix by rows use \code{byrow = TRUE}
\begin{knitrout}\footnotesize
\definecolor{shadecolor}{rgb}{0.969, 0.969, 0.969}\color{fgcolor}\begin{kframe}
\begin{alltt}
\hlcom{# vector to matrix (by rows)}
\hlstd{b} \hlkwb{<-} \hlnum{1}\hlopt{:}\hlnum{8}

\hlstd{B} \hlkwb{<-} \hlkwd{matrix}\hlstd{(b,} \hlkwc{nrow} \hlstd{=} \hlnum{2}\hlstd{,} \hlkwc{ncol} \hlstd{=} \hlnum{4}\hlstd{,} \hlkwc{byrow} \hlstd{=} \hlnum{TRUE}\hlstd{)}
\hlstd{B}
\end{alltt}
\begin{verbatim}
##      [,1] [,2] [,3] [,4]
## [1,]    1    2    3    4
## [2,]    5    6    7    8
\end{verbatim}
\end{kframe}
\end{knitrout}

\end{frame}

%------------------------------------------------

\begin{frame}
\frametitle{More about R Matrices}

\bbi
  \item R stores matrices (and arrays in general) as vectors.
  \item Which means that matrices are also \textbf{atomic} structures.
  \item Matrices in R are stored column-major (i.e. by columns).
  \item This is like Fortran, Matlab, and Julia, but not like C or Python (e.g. numpy).
\ei

\end{frame}

%------------------------------------------------

\begin{frame}[fragile]
\frametitle{Matrix: Column-major}

\begin{knitrout}\footnotesize
\definecolor{shadecolor}{rgb}{0.969, 0.969, 0.969}\color{fgcolor}\begin{kframe}
\begin{alltt}
\hlkwd{as.vector}\hlstd{(A)}
\end{alltt}
\begin{verbatim}
## [1] 1 2 3 4 5 6 7 8
\end{verbatim}
\begin{alltt}
\hlkwd{as.vector}\hlstd{(B)}
\end{alltt}
\begin{verbatim}
## [1] 1 5 2 6 3 7 4 8
\end{verbatim}
\end{kframe}
\end{knitrout}

\end{frame}

%------------------------------------------------

\begin{frame}
\frametitle{Matrices and Vectors in R}

It is important to distinguish vectors and matrices, especially in R:
\bi
  \item In matrix algebra we use the convention that vectors are column vectors 
  (i.e. they are $n \times 1$ matrices).
  \item In R, a \code{vector} with $n$ elements is not the same as an $n \times 1$ \code{matrix}.
  \item Vectors in R behave more like ``row vectors'' (especially when displayed).
  \item However, depending on the type of functions you apply to vectors, sometimes R will handle vectors like if they were column vectors.
\ei

\end{frame}

%------------------------------------------------

\begin{frame}[fragile]
\frametitle{R common data structures}
\begin{center}
\ig[width=8cm]{images/data_structures.pdf}
\end{center}
\end{frame}

%------------------------------------------------

\begin{frame}
\frametitle{Matrices vs Dataframes in R}

It's also important to distinguish between matrices and data.frames:
\bi
  \item Both objects allow us to store data in a 2-dimensional object.
  \item In many cases, both R matrices and data.frames have similar behaviors.
  \item This is mostly the case when they are displayed on the screen.
  \item And sometimes it is hard to distinguish between a matrix and a data.frame
  by just looking at the displayed content on the screen.
\ei

\end{frame}

%------------------------------------------------

\begin{frame}
\frametitle{Matrices vs Dataframes in R}

\bi
  \item Dataframes are actually lists.
  \item Internally, a data.frame is stored as an R list (typically a list of vectors).
  \item This provides a very flexible way to manipulate data.frames, using 
  the \code{\$} operator, double brackets\code{[[]]}, and the regular \code{[,]}
  \item An R matrix is internally stored as a vector (with a \code{dim} attribute).
  \item Which means that you use the \code{[,]} operator.
\ei

\end{frame}

%------------------------------------------------

\begin{frame}
\frametitle{Matrices vs Dataframes in R}

Both matrices and data.frames have common methods

\bi
  \item bracket notation: {\mdlit \code{[ , ]}}
  \item dimensions: {\mdlit \code{dim()}}
  \item number of rows: {\mdlit \code{nrow()}}
  \item number of columns: {\mdlit \code{ncol()}}
  \item dim names: {\mdlit \code{dimnames()}}
  \item names of columns: {\mdlit \code{colnames()}}
  \item names of rows: {\mdlit \code{rownames()}}
  \item apply functions: {\mdlit \code{apply()}}
\ei

\end{frame}

%------------------------------------------------

\begin{frame}
\frametitle{Notation System}

\bb{Notation system to extract values from R objects}
\bbi
  \item to extract values use brackets: \code{[ ]}
  \item inside the brackets specify indices
  \item use as many indices as dimensions in the object
  \item each index is separated by comma
  \item indices can be numbers, logicals, or names
\ei
\eb
\end{frame}

%------------------------------------------------

\begin{frame}
\frametitle{Elements-Extraction Notation System}

\begin{center}
 \begin{tabular}{l l l}
  \hline
   object & notation & example \\
  \hline
  vector & \code{[ ]} & \code{v[1:5]} \\  
  factor & \code{[ ]} & \code{g[1:5]} \\  
  matrix & \code{[ , ]} & \code{m[1:5, 1:3]} \\  
  array & \code{[ , , ]} & \code{arr[1, 2, 3]} \\  
        & \code{[ , , , ]} & \code{arr[1, 2, 3, 4]} \\  
  list & \code{[ ]} & \code{lst[3]} \\
       & \code{[[ ]]} & \code{lst[[3]]} \\
       & \code{\$} & \code{lst\$name} \\
  data frame & \code{[ , ]} & \code{df[1, 2]} \\
             & \code{\$} & \code{df\$name} \\
  \hline
 \end{tabular}
\end{center}

\end{frame}

%------------------------------------------------

\begin{frame}[fragile]
\frametitle{Brackets, Parentheses, and Braces in R}
\begin{center}
\ig[width=9cm]{images/braces.pdf}
\end{center}
\end{frame}

%------------------------------------------------

\begin{frame}
\begin{center}
\Huge{\hilit{Basic Matrix Operations}}
\end{center}
\end{frame}

%------------------------------------------------

\begin{frame}
\frametitle{Matrix Operations in R}

We first quickly go through the basic matrix operations in R

\bi
  \item transpose
  \item addition
  \item scalar multiplication
  \item matrix-vector multiplication
  \item matrix-matrix multiplication
\ei

\end{frame}

%------------------------------------------------

\begin{frame}[fragile]
\frametitle{Matrix Transpose}

The transpose of a $n \times p$ matrix $\mathbf{X}$ is the $p \times n$ matrix 
$\mathbf{X}^{\mathsf{T}}$. In R the transpose is given by the function \code{t()}

\begin{knitrout}\footnotesize
\definecolor{shadecolor}{rgb}{0.969, 0.969, 0.969}\color{fgcolor}\begin{kframe}
\begin{alltt}
\hlcom{# matrix X}
\hlstd{X} \hlkwb{<-} \hlkwd{matrix}\hlstd{(}\hlnum{1}\hlopt{:}\hlnum{6}\hlstd{,} \hlnum{2}\hlstd{,} \hlnum{3}\hlstd{)}
\hlstd{X}
\end{alltt}
\begin{verbatim}
##      [,1] [,2] [,3]
## [1,]    1    3    5
## [2,]    2    4    6
\end{verbatim}
\begin{alltt}
\hlcom{# transpose of X}
\hlkwd{t}\hlstd{(X)}
\end{alltt}
\begin{verbatim}
##      [,1] [,2]
## [1,]    1    2
## [2,]    3    4
## [3,]    5    6
\end{verbatim}
\end{kframe}
\end{knitrout}

\end{frame}

%------------------------------------------------

\begin{frame}[fragile]
\frametitle{Matrix Addition}

Matrix addition of two matrices $\mathbf{A} + \mathbf{B}$ is defined when 
$\mathbf{A}$ and $\mathbf{B}$ have the same dimensions:

\begin{knitrout}\footnotesize
\definecolor{shadecolor}{rgb}{0.969, 0.969, 0.969}\color{fgcolor}\begin{kframe}
\begin{alltt}
\hlstd{A} \hlkwb{<-} \hlkwd{matrix}\hlstd{(}\hlnum{1}\hlopt{:}\hlnum{6}\hlstd{,} \hlnum{2}\hlstd{,} \hlnum{3}\hlstd{)}
\hlstd{B} \hlkwb{<-} \hlkwd{matrix}\hlstd{(}\hlnum{7}\hlopt{:}\hlnum{9}\hlstd{,} \hlnum{2}\hlstd{,} \hlnum{3}\hlstd{)}
\hlstd{A} \hlopt{+} \hlstd{B}
\end{alltt}
\begin{verbatim}
##      [,1] [,2] [,3]
## [1,]    8   12   13
## [2,]   10   11   15
\end{verbatim}
\end{kframe}
\end{knitrout}

\end{frame}

%------------------------------------------------

\begin{frame}[fragile]
\frametitle{Scalar Multiplication}

We can multiply a matrix by a scalar using the usual product operator \code{*}, 
moreover it doesn't matter if we pre-multiply or post-multiply:

\begin{knitrout}\footnotesize
\definecolor{shadecolor}{rgb}{0.969, 0.969, 0.969}\color{fgcolor}\begin{kframe}
\begin{alltt}
\hlstd{X} \hlkwb{<-} \hlkwd{matrix}\hlstd{(}\hlnum{1}\hlopt{:}\hlnum{3}\hlstd{,} \hlnum{3}\hlstd{,} \hlnum{4}\hlstd{)}

\hlcom{# (pre)multiply X by 0.5}
\hlstd{(}\hlnum{1}\hlopt{/}\hlnum{2}\hlstd{)} \hlopt{*} \hlstd{X}
\end{alltt}
\begin{verbatim}
##      [,1] [,2] [,3] [,4]
## [1,]  0.5  0.5  0.5  0.5
## [2,]  1.0  1.0  1.0  1.0
## [3,]  1.5  1.5  1.5  1.5
\end{verbatim}
\end{kframe}
\end{knitrout}

\end{frame}

%------------------------------------------------

\begin{frame}[fragile]
\frametitle{Scalar Multiplication}

You can also postmultiply by a scalar (although this is not recommended 
because may confuse readers):

\begin{knitrout}\footnotesize
\definecolor{shadecolor}{rgb}{0.969, 0.969, 0.969}\color{fgcolor}\begin{kframe}
\begin{alltt}
\hlstd{X} \hlkwb{<-} \hlkwd{matrix}\hlstd{(}\hlnum{1}\hlopt{:}\hlnum{3}\hlstd{,} \hlnum{3}\hlstd{,} \hlnum{4}\hlstd{)}

\hlcom{# (post)multiply X by 0.5}
\hlstd{X} \hlopt{*} \hlstd{(}\hlnum{1}\hlopt{/}\hlnum{2}\hlstd{)}
\end{alltt}
\begin{verbatim}
##      [,1] [,2] [,3] [,4]
## [1,]  0.5  0.5  0.5  0.5
## [2,]  1.0  1.0  1.0  1.0
## [3,]  1.5  1.5  1.5  1.5
\end{verbatim}
\end{kframe}
\end{knitrout}

\end{frame}

%------------------------------------------------

\begin{frame}[fragile]
\frametitle{Matrix-Matrix Multiplication}

The matrix product operator in R is {\hilit \code{\%*\%}}. 
We can multiply matrices $\mathbf{A}$ and $\mathbf{B}$ if the number of columns 
of $\mathbf{A}$ is equal to the number of rows of $\mathbf{B}$

\begin{knitrout}\footnotesize
\definecolor{shadecolor}{rgb}{0.969, 0.969, 0.969}\color{fgcolor}\begin{kframe}
\begin{alltt}
\hlstd{A} \hlkwb{<-} \hlkwd{matrix}\hlstd{(}\hlnum{1}\hlopt{:}\hlnum{6}\hlstd{,} \hlnum{2}\hlstd{,} \hlnum{3}\hlstd{)}
\hlstd{B} \hlkwb{<-} \hlkwd{matrix}\hlstd{(}\hlnum{7}\hlopt{:}\hlnum{9}\hlstd{,} \hlnum{3}\hlstd{,} \hlnum{2}\hlstd{)}

\hlstd{A} \hlopt \hlstd{B}
\end{alltt}
\begin{verbatim}
##      [,1] [,2]
## [1,]   76   76
## [2,]  100  100
\end{verbatim}
\end{kframe}
\end{knitrout}

\end{frame}

%------------------------------------------------

\begin{frame}[fragile]
\frametitle{Matrix-Matrix Multiplication}

We can multiply matrices $\mathbf{A}$ and $\mathbf{B}$ if the number of columns 
of $\mathbf{A}$ is equal to the number of rows of $\mathbf{B}$

\begin{knitrout}\footnotesize
\definecolor{shadecolor}{rgb}{0.969, 0.969, 0.969}\color{fgcolor}\begin{kframe}
\begin{alltt}
\hlstd{A} \hlkwb{<-} \hlkwd{matrix}\hlstd{(}\hlnum{1}\hlopt{:}\hlnum{6}\hlstd{,} \hlnum{2}\hlstd{,} \hlnum{3}\hlstd{)}
\hlstd{B} \hlkwb{<-} \hlkwd{matrix}\hlstd{(}\hlnum{7}\hlopt{:}\hlnum{9}\hlstd{,} \hlnum{3}\hlstd{,} \hlnum{2}\hlstd{)}

\hlstd{B} \hlopt \hlstd{A}
\end{alltt}
\begin{verbatim}
##      [,1] [,2] [,3]
## [1,]   21   49   77
## [2,]   24   56   88
## [3,]   27   63   99
\end{verbatim}
\end{kframe}
\end{knitrout}

\end{frame}

%------------------------------------------------

\begin{frame}[fragile]
\frametitle{Cross-Products}

A very common type of products in multivariate data analysis are 
$\mathbf{X^{\mathsf{T}} X}$ and $\mathbf{X X^{\mathsf{T}}}$, sometimes known as 
cross-products:

\begin{columns}[t]
\begin{column}{0.45\textwidth}

\begin{knitrout}\footnotesize
\definecolor{shadecolor}{rgb}{0.969, 0.969, 0.969}\color{fgcolor}\begin{kframe}
\begin{alltt}
\hlcom{# output with return()}
\hlcom{# cross-product}
\hlkwd{t}\hlstd{(A)} \hlopt \hlstd{A}
\end{alltt}
\begin{verbatim}
##      [,1] [,2] [,3]
## [1,]    5   11   17
## [2,]   11   25   39
## [3,]   17   39   61
\end{verbatim}
\end{kframe}
\end{knitrout}
\end{column}

\begin{column}{0.45\textwidth}
\begin{knitrout}\footnotesize
\definecolor{shadecolor}{rgb}{0.969, 0.969, 0.969}\color{fgcolor}\begin{kframe}
\begin{alltt}
\hlcom{# cross-product}
\hlstd{A} \hlopt \hlkwd{t}\hlstd{(A)}
\end{alltt}
\begin{verbatim}
##      [,1] [,2]
## [1,]   35   44
## [2,]   44   56
\end{verbatim}
\end{kframe}
\end{knitrout}
\end{column}
\end{columns}

\end{frame}

%------------------------------------------------

\begin{frame}[fragile]
\frametitle{Cross-Products}

R provides functions {\hilit \code{crossprod()}} and {\hilit \code{tcrossprod()}} 
which are formally equivalent to:

\bbi
  \item \code{crossprod(X, X)} $\equiv$ \code{t(X) \%*\% X}
  \item \code{tcrossprod(X, X)} $\equiv$ \code{X \%*\% t(X)}
\ei

However, \code{crossprod()} and \code{tcrossprod()} are usually faster than using 
\code{t()} and \code{\%*\%}


\end{frame}

%------------------------------------------------

\begin{frame}[fragile]
\frametitle{Matrix-Vector Multiplication}

We can post-multiply an $n \times p$ matrix $\mathbf{X}$ with a vector 
$\mathbf{b}$ with $p$ elements. This means making linear combinations 
(weighted sums) of the columns of $\mathbf{X}$:

\begin{knitrout}\footnotesize
\definecolor{shadecolor}{rgb}{0.969, 0.969, 0.969}\color{fgcolor}\begin{kframe}
\begin{alltt}
\hlstd{X} \hlkwb{<-} \hlkwd{matrix}\hlstd{(}\hlnum{1}\hlopt{:}\hlnum{12}\hlstd{,} \hlnum{3}\hlstd{,} \hlnum{4}\hlstd{)}
\hlstd{b} \hlkwb{<-} \hlkwd{seq}\hlstd{(}\hlnum{0.25}\hlstd{,} \hlnum{1}\hlstd{,} \hlkwc{by} \hlstd{=} \hlnum{0.25}\hlstd{)}
\hlstd{X} \hlopt \hlstd{b}
\end{alltt}
\begin{verbatim}
##      [,1]
## [1,] 17.5
## [2,] 20.0
## [3,] 22.5
\end{verbatim}
\end{kframe}
\end{knitrout}

\end{frame}

%------------------------------------------------

\begin{frame}[fragile]
\frametitle{Vector-Matrix Multiplication}

In R we can pre-multiply a vector $\mathbf{a}$ (with $n$ elements) with an $n \times p$ 
matrix $\mathbf{X}$. This means making linear combinations (weighted sums) of 
the rows of $\mathbf{X}$:

\begin{knitrout}\footnotesize
\definecolor{shadecolor}{rgb}{0.969, 0.969, 0.969}\color{fgcolor}\begin{kframe}
\begin{alltt}
\hlstd{X} \hlkwb{<-} \hlkwd{matrix}\hlstd{(}\hlnum{1}\hlopt{:}\hlnum{12}\hlstd{,} \hlnum{3}\hlstd{,} \hlnum{4}\hlstd{)}
\hlstd{a} \hlkwb{<-} \hlnum{1}\hlopt{:}\hlnum{3}
\hlstd{a} \hlopt \hlstd{X}
\end{alltt}
\begin{verbatim}
##      [,1] [,2] [,3] [,4]
## [1,]   14   32   50   68
\end{verbatim}
\end{kframe}
\end{knitrout}

\end{frame}

%------------------------------------------------

\begin{frame}[fragile]
\frametitle{Note about \code{\%*\%}}

Notice that when we use the product operator \code{\%*\%}, R is smart enough to use 
the convention that vectors are $n \times 1$ matrices.

Notice also that if we ask for a vector-matrix multiplication, we can use both
formulas:

\bi
  \item \code{a \%*\% X}
  \item \code{t(a) \%*\% X}
\ei

R will reformat the $n$ vector as an $n \times 1$ matrix first.

\end{frame}

%------------------------------------------------

\begin{frame}
\frametitle{Other Functions}

Here are some other interesting functions for matrices

\bi
  \item {\mdlit \code{det()}}: determinant
  \item {\mdlit \code{diag()}}: extract or replace the diagonal elements
  \item {\mdlit \code{solve()}}: solve system of equations
  \item {\mdlit \code{svd()}}: singular value decomposition
  \item {\mdlit \code{eigen()}}: eigen-decomposition
  \item {\mdlit \code{qr()}}: QR decomposition
\ei

\end{frame}

%------------------------------------------------

\end{document}
