\documentclass[12pt]{beamer}\usepackage[]{graphicx}\usepackage[]{color}
%% maxwidth is the original width if it is less than linewidth
%% otherwise use linewidth (to make sure the graphics do not exceed the margin)
\makeatletter
\def\maxwidth{ %
  \ifdim\Gin@nat@width>\linewidth
    \linewidth
  \else
    \Gin@nat@width
  \fi
}
\makeatother

\definecolor{fgcolor}{rgb}{0.345, 0.345, 0.345}
\newcommand{\hlnum}[1]{\textcolor[rgb]{0.686,0.059,0.569}{#1}}%
\newcommand{\hlstr}[1]{\textcolor[rgb]{0.192,0.494,0.8}{#1}}%
\newcommand{\hlcom}[1]{\textcolor[rgb]{0.678,0.584,0.686}{\textit{#1}}}%
\newcommand{\hlopt}[1]{\textcolor[rgb]{0,0,0}{#1}}%
\newcommand{\hlstd}[1]{\textcolor[rgb]{0.345,0.345,0.345}{#1}}%
\newcommand{\hlkwa}[1]{\textcolor[rgb]{0.161,0.373,0.58}{\textbf{#1}}}%
\newcommand{\hlkwb}[1]{\textcolor[rgb]{0.69,0.353,0.396}{#1}}%
\newcommand{\hlkwc}[1]{\textcolor[rgb]{0.333,0.667,0.333}{#1}}%
\newcommand{\hlkwd}[1]{\textcolor[rgb]{0.737,0.353,0.396}{\textbf{#1}}}%
\let\hlipl\hlkwb

\usepackage{framed}
\makeatletter
\newenvironment{kframe}{%
 \def\at@end@of@kframe{}%
 \ifinner\ifhmode%
  \def\at@end@of@kframe{\end{minipage}}%
  \begin{minipage}{\columnwidth}%
 \fi\fi%
 \def\FrameCommand##1{\hskip\@totalleftmargin \hskip-\fboxsep
 \colorbox{shadecolor}{##1}\hskip-\fboxsep
     % There is no \\@totalrightmargin, so:
     \hskip-\linewidth \hskip-\@totalleftmargin \hskip\columnwidth}%
 \MakeFramed {\advance\hsize-\width
   \@totalleftmargin\z@ \linewidth\hsize
   \@setminipage}}%
 {\par\unskip\endMakeFramed%
 \at@end@of@kframe}
\makeatother

\definecolor{shadecolor}{rgb}{.97, .97, .97}
\definecolor{messagecolor}{rgb}{0, 0, 0}
\definecolor{warningcolor}{rgb}{1, 0, 1}
\definecolor{errorcolor}{rgb}{1, 0, 0}
\newenvironment{knitrout}{}{} % an empty environment to be redefined in TeX

\usepackage{alltt}
\usepackage{tikz}

% make it pretty
% get rid of junk
\usetheme{default}
\usefonttheme[onlymath]{serif}
\beamertemplatenavigationsymbolsempty

% define a bunch of colors
\definecolor{offwhite}{RGB}{255,250,240}
\definecolor{gray}{RGB}{155,155,155}
\definecolor{foreground}{RGB}{80,80,80}
\definecolor{background}{RGB}{255,255,255}
%\definecolor{title}{RGB}{255,199,0}
\definecolor{title}{RGB}{89,132,212}
%\definecolor{subtitle}{RGB}{89,132,212}
\definecolor{subtitle}{RGB}{255,199,0}
\definecolor{hilit}{RGB}{248,117,79}
\definecolor{vhilit}{RGB}{255,111,207}
\definecolor{lolit}{RGB}{200,200,200}
\definecolor{lit}{RGB}{255,199,0}
\definecolor{mdlit}{RGB}{89,132,212}
\definecolor{link}{RGB}{248,117,79}

% a few color macros
\newcommand{\hilit}{\color{hilit}}
\newcommand{\vhilit}{\color{vhilit}}
\newcommand{\lit}{\color{lit}}
\newcommand{\mdlit}{\color{mdlit}}
\newcommand{\lolit}{\color{lolit}}

% use those colors
\setbeamercolor{titlelike}{fg=title}
\setbeamercolor{subtitle}{fg=subtitle}
\setbeamercolor{frametitle}{fg=gray}
%\setbeamercolor{structure}{fg=subtitle}
\setbeamercolor{structure}{fg=title}
\setbeamercolor{institute}{fg=lolit}
\setbeamercolor{normal text}{fg=foreground,bg=background}
\setbeamertemplate{itemize subitem}{{\textendash}}
\setbeamerfont{itemize/enumerate subbody}{size=\small}
\setbeamerfont{itemize/enumerate subitem}{size=\small}

% center title of slides
\setbeamertemplate{blocks}[rounded]
\setbeamertemplate{frametitle}[default][center]

% page number
\setbeamerfont{page number in foot}{size=\footnotesize}
\setbeamertemplate{footline}[frame number]

% default link color
\hypersetup{colorlinks, urlcolor={link}}

% a few macros
\newcommand{\code}[1]{\texttt{#1}}
\newcommand{\hicode}[1]{{\hilit \texttt{#1}}}
\newcommand{\locode}[1]{{\lolit \texttt{#1}}}
\newcommand{\bb}[1]{\begin{block}{#1}}
\newcommand{\eb}{\end{block}}
\newcommand{\bi}{\begin{itemize}}
\newcommand{\bbi}{\vspace{4pt} \begin{itemize} \itemsep8pt}
\newcommand{\ei}{\end{itemize}}
\newcommand{\bv}{\begin{verbatim}}
\newcommand{\ev}{\end{verbatim}}
\newcommand{\ig}{\includegraphics}
\newcommand{\subt}[1]{{\footnotesize \color{subtitle} {#1}}}
\newcommand{\ttsm}{\tt \small}
\newcommand{\ttfn}{\tt \footnotesize}
\newcommand{\figh}[2]{\centerline{\includegraphics[height=#2\textheight]{#1}}}
\newcommand{\figw}[2]{\centerline{\includegraphics[width=#2\textwidth]{#1}}}



%------------------------------------------------

\title{Statistical Operations and Matrices (I)}
\subtitle{Matrix Algebra 4 Statistical Learning}
\author{\href{http://www.gastonsanchez.com}{Gaston Sanchez}}
\institute{\href{https://creativecommons.org/licenses/by-sa/4.0/}{\tt \scriptsize \color{foreground} CC BY-SA 4.0}}
\date{}
\IfFileExists{upquote.sty}{\usepackage{upquote}}{}
\begin{document}



% no page number in first slide
{
  \setbeamertemplate{footline}{} 
  \frame{\titlepage} 
}

%------------------------------------------------

\begin{frame}
\begin{center}
\Huge{\hilit{Vector-Matrix Notation \\ for Statistical Operations}}
\end{center}
\end{frame}

%------------------------------------------------

\begin{frame}
\frametitle{Motivation}

I want to discuss how we can use vector-matrix notation to represent
some basic statistical operations and summaries.

\bigskip
First we need to quickly review some concepts around inner products.

\end{frame}

%------------------------------------------------

\begin{frame}
\frametitle{Inner Product}

The inner product of two vectors $\mathbf{x}$ and $\mathbf{y}$---of the same size---is defined as:

{\Large
$$
\langle \mathbf{x}, \mathbf{y} \rangle = \mathbf{x^{\mathsf{T}} y} = \sum_i x_i y_i
$$
}

basically the inner product consists of the element-by-element product of 
$\mathbf{x}$ and $\mathbf{y}$, and then adding everything up. The result is not 
another vector but a single number, a scalar. 

\end{frame}

%------------------------------------------------

\begin{frame}
\frametitle{Inner Product}

We can also write the inner product $\langle \mathbf{x}, \mathbf{y} \rangle$ in 
vector notation as:

$$
\mathbf{x^\mathsf{T} y} = (x_1 \dots x_n) 
\begin{pmatrix} 
y_1 \\
\vdots \\
y_n
\end{pmatrix}
= \sum_{i = 1}^{n} x_i y_i
$$

\end{frame}

%------------------------------------------------

\begin{frame}
\frametitle{Inner Product}

Keep in mind that inner products can be generalized to any type of 
metric $\mathbf{M}$ matrix:

{\Large
$$
\langle \mathbf{x}, \mathbf{y} \rangle_{M} = \mathbf{x^{\mathsf{T}} M y}
$$
}

\end{frame}

%------------------------------------------------

\begin{frame}
\frametitle{Inner Product}

Having an inner space endowed with an inner product, we can derive other concepts such as:

\bbi
  \item \textbf{Length} of a vector (and \textit{norms} in general)
  \item \textbf{Distance} between points
  \item \textbf{Angle} between vectors
  \item \textbf{Projection} of vectors
\ei

\end{frame}

%------------------------------------------------

\begin{frame}
\frametitle{Length}

Another important usage of the inner product is that it allows us to define the 
\textbf{length} of a vector $\mathbf{x}$ denoted by:

{\Large
$$
\| \mathbf{x} \| = \sqrt{\mathbf{x^\mathsf{T} x}}
$$
}

which is typically known as the (Euclidean) \textbf{norm} of a vector. \\
{\lolit (There are actually other types of norms)}

\end{frame}

%------------------------------------------------

\begin{frame}
\frametitle{Length}

The inner product of a vector with itself is equal to 
its squared norm: $\mathbf{x^\mathsf{T} x} = \| \mathbf{x} \|^2$

{\Large
$$
\| \mathbf{x} \|^2 = \sum_{i=1}^{n} x_{i}^{2}
$$
}

\end{frame}

%------------------------------------------------

\begin{frame}
\frametitle{Distance}

The square (Eculidean) distance between two vectors $\mathbf{x}$ and $\mathbf{y}$ 
can be obtained as:

{\large
$$
d^{2} (\mathbf{x}, \mathbf{y}) = \| \mathbf{x} - \mathbf{y} \|^2 = 
(\mathbf{x-y})^{\mathsf{T}}(\mathbf{x-y})
$$
}

\pause
\bigskip
The distance between two vectors $\mathbf{x}$ and $\mathbf{y}$ can be obtained as:

{\large
$$
d(\mathbf{x}, \mathbf{y}) = \| \mathbf{x} - \mathbf{y} \| = \sqrt{(\mathbf{x-y})^{\mathsf{T}}(\mathbf{x-y})}
$$
}

\end{frame}

%------------------------------------------------

\begin{frame}
\frametitle{Angle between Vectors}

The \textbf{angle} $\theta$ between two nonzero vectors 
$\mathbf{x, y}$ can also be expressed using inner products:

{\Large
$$
cos(\theta) = \frac{\mathbf{x^\mathsf{T} y}}{\sqrt{\mathbf{x^\mathsf{T} x}} \hspace{1mm} \sqrt{\mathbf{y^\mathsf{T} y}}}
= \frac{\mathbf{x^\mathsf{T} y}}{\| \mathbf{x} \| \hspace{1mm} \| \mathbf{y} \|}
$$
}

\end{frame}

%------------------------------------------------

\begin{frame}
\frametitle{Angle}

Equivalently, we can reexpress the formula of the inner product using 

{\Large
$$
\mathbf{x^\mathsf{T} y} = \| \mathbf{x} \| \hspace{1mm} \| \mathbf{y} \| \hspace{1mm} cos(\theta)
$$
}

\end{frame}

%------------------------------------------------

\begin{frame}
\frametitle{Orthogonality}

Besides calculating lengths of vectors and angles between vectors, an inner 
product allows us to know whether two vectors are orthogonal. 

\bigskip
In a two dimensional space, orthogonality is equivalent to perpendicularity; 
so if two vectors are perpendicular to each 
other---the angle between them is a 90 degree angle---they are orthogonal. 


\bigskip
Two vectors vectors $\mathbf{x}$ and $\mathbf{y}$ 
are orthogonal if their inner product is zero:

$$
\mathbf{x^\mathsf{T} y} = 0 \quad \iff \quad \mathbf{x} \hspace{1mm} \bot \hspace{1mm} \mathbf{y} 
$$

\end{frame}

%------------------------------------------------

\begin{frame}
\frametitle{Projection}

The last aspect I want to touch related with the inner product is the so-called 
projections. More specifically: orthogonal projection of a 
vector $\mathbf{y}$ onto another vector $\mathbf{x}$. 

\bigskip
The basic notion of projection requires two ingredients: two vectors, 
$\mathbf{x}$ and $\mathbf{y}$. To obtain the projection of $\mathbf{y}$ onto 
$\mathbf{x}$, we need to express $\mathbf{x}$ in unit norm.

\end{frame}

%------------------------------------------------

\begin{frame}
\frametitle{Unit Vector}

Recall that a \textbf{unit vector} is a vector of length 1, sometimes 
also called a \textit{direction} or \textit{unitary} vector.

\bigskip
In other words, if $\mathbf{v}$ is a unit vector, this means that $\| \mathbf{v} \| = 1$. 

\bigskip
{\lit Given a non-zero vector $\mathbf{v}$, how do we get a unit vector?}

\end{frame}

%------------------------------------------------

\begin{frame}
\frametitle{Unit Vector}

Given a non-zero vector $\mathbf{v}$, you can get a unit vector $\mathbf{u}$ by dividing 
$\mathbf{v}$ by its norm, that is:

{\Large
$$
\mathbf{u} = \frac{\mathbf{v}}{\| \mathsf{v} \|} = \frac{\mathbf{v}}{\mathbf{\sqrt{v^{\mathsf{T}} v}}}
$$
}

Some say that we ``normalize $\mathbf{v}$''

\end{frame}

%------------------------------------------------

\begin{frame}
\frametitle{Projection}

Having two nonzero 
vectors $\mathbf{x}$ and $\mathbf{y}$, we can project $\mathbf{y}$ on 
$\mathbf{x}$, denoted by $\hat{\mathbf{y}}$ as:

{\Large
$$
\hat{\mathbf{y}} = \mathbf{x} \left( \frac{\mathbf{y^\mathsf{T} x}}{\mathbf{x^\mathsf{T} x}} \right)
$$
}

Note that the term in parenthesis is just a scalar.

\end{frame}

%------------------------------------------------

\begin{frame}
\frametitle{Projection}

Note that

{\large
$$
\hat{\mathbf{y}} = \mathbf{x} \left( \frac{\mathbf{y^\mathsf{T} x}}{\mathbf{x^\mathsf{T} x}} \right) = \mathbf{x} \left( \frac{\mathbf{x^\mathsf{T} y}}{\mathbf{x^\mathsf{T} x}} \right)
$$
}

can also be written as:

{\large
$$
\hat{\mathbf{y}} = \mathbf{x} (\mathbf{x^\mathsf{T} x})^{-1} \mathbf{x^\mathsf{T} y} 
$$
}

\begin{center}
{\lit Does it look familiar?}
\end{center}

\end{frame}

%------------------------------------------------

\begin{frame}
\frametitle{Orthogonal Projection}
\begin{center}
\ig[width=8cm]{images/orthogonal-projection.png}
\end{center}
\end{frame}

%------------------------------------------------

\begin{frame}
\frametitle{Projection}

We can actually express $\hat{\mathbf{y}}$ as $a \mathbf{x}$. This means that a projection implies 
multiplying $\mathbf{x}$ by some number $a$, such that
$\hat{\mathbf{y}} = a \mathbf{x}$ is a stretched or shrinked version of $\mathbf{x}$.

\end{frame}

%------------------------------------------------

\begin{frame}
\frametitle{Inner Products ... So what?}

We'll see in a moment how many statistical summaries can be represented 
with inner products.

\end{frame}

%------------------------------------------------

\begin{frame}
\begin{center}
\Huge{\hilit{Review of Some \\ Statistical Operations}}
\end{center}
\end{frame}

%------------------------------------------------

\begin{frame}
\frametitle{Statistical Summaries}

I want to review basic statistical summaries and see how we can express them
in vector-matrix notation:

\bbi
  \item Sum of values
  \item Sum of squared values
  \item Mean
  \item Variance  
\ei

\end{frame}

%------------------------------------------------

\begin{frame}
\frametitle{Sum of Values}

Consider a variable $X \in \mathbb{R}^n$ represented with a vector 
$\mathbf{x} = (x_1, x_2, \dots, x_n)$

\bigskip
A common operation consists of adding the elements of the vector
$$
\sum_{i=1}^{n} x_i
$$

\pause
\bigskip
This sum can be expressed in vector notation as:
$$
\sum_{i=1}^{n} x_i = \mathbf{x^{\mathsf{T}} 1} = \mathbf{1^{\mathsf{T}} x}
$$
where $\mathbf{1}$ is a vector of $n$ elements equal to 1

\end{frame}

%------------------------------------------------

\begin{frame}
\frametitle{Mean}

The (arithmetic) mean of $X$ is
$$
\bar{x} = \frac{1}{n} \sum_{i=1}^{n} x_i
$$

\pause
\bigskip
Using vector notation the mean is expressed as:
$$
\bar{x} = \frac{1}{n} \mathbf{x^{\mathsf{T}} 1}
$$

\end{frame}

%------------------------------------------------

\begin{frame}
\frametitle{Sum of Square dValues}

Another common operation is the sum of squared values
$$
\sum_{i=1}^{n} x_{i}^{2}
$$

\pause
\bigskip
Using vector notation this sum can be written as an inner product:
$$
\sum_{i=1}^{n} x_{i}^{2} = \mathbf{x^{\mathsf{T}} x} = \| \mathsf{x} \|^2
$$

\end{frame}

%------------------------------------------------

\begin{frame}
\frametitle{Variance}

The variance of $X$ {\lolit (in its population version)}
$$
var(X) = \frac{1}{n} \sum_{i=1}^{n} (x_i - \bar{x})^2
$$

\bi
  \item Let $\mathbf{\bar{x}}$ be an $n$-element vector constant of $\bar{x}$ values
  \item Let $\mathbf{\tilde{x}} = \mathbf{x} - \mathbf{\bar{x}}$
\ei

\pause
\bigskip
The variance can be expressed as:
$$
var(X) = \frac{1}{n} \mathbf{\tilde{x}^{\mathsf{T}} \tilde{x}}
$$

\end{frame}

%------------------------------------------------

\begin{frame}
\frametitle{Variance}

If you consider the ``sample'' variance
$$
var(X) = \frac{1}{n-1} \sum_{i=1}^{n} (x_i - \bar{x})^2
$$

\bigskip
Then:
$$
var(X) = \frac{1}{n-1} \mathbf{\tilde{x}^{\mathsf{T}} \tilde{x}}
$$

\end{frame}

%------------------------------------------------

\begin{frame}
\frametitle{Variance}

\bi
  \item I will sometimes consider $\mathbf{x}$ to be mean-centered.
  \item This means that $\bar{x}$ has already been subtracted from each element $x_i$
\ei

In this case the variance can be compactly expressed as:
$$
var(X) = \frac{1}{n} \mathbf{x^{\mathsf{T}} x}
$$

\end{frame}

%------------------------------------------------

\begin{frame}
\frametitle{Covariance}

Consider two variables $X$ and $Y$. The covariance $cov(X,Y)$
$$
cov(X,Y) = \frac{1}{n} \sum_{i=1}^{n} (x_i - \bar{x})(y_i - \bar{y})
$$

\pause
Assuming that $X$ and $Y$ are mean-centered, and represented by 
$\mathbf{x}$ and $\mathbf{y}$, respectively, the covariance can be expressed as:

{\large
$$
cov(\mathbf{x}, \mathbf{y}) = \frac{1}{n} \mathbf{x^{\mathsf{T}} y}
$$
}

\end{frame}

%------------------------------------------------

\begin{frame}
\frametitle{Correlation}

Consider two variables $X$ and $Y$. The correlation $cor(X,Y)$
$$
cor(X,Y) = \frac{cov(X,Y)}{\sqrt{var(X)} \sqrt{var(Y)}}
$$

\pause
Assuming that $X$ and $Y$ are mean-centered, and represented by 
$\mathbf{x}$ and $\mathbf{y}$, respectively, the covariance can be expressed as:
$$
cor(\mathbf{x}, \mathbf{y}) = \frac{\mathbf{x^{\mathsf{T}} y}}{\sqrt{\mathbf{x^\mathsf{T} x}} \sqrt{\mathbf{y^\mathsf{T} y}}} = \frac{\mathbf{x^{\mathsf{T}} y}}{\| \mathbf{x} \| \hspace{1mm} \| \mathbf{y} \|}
$$

\end{frame}

%------------------------------------------------

\begin{frame}
\frametitle{Correlation: Geometric Interpretation}

What does it mean geometrically:
$$
cor(\mathbf{x}, \mathbf{y}) = \frac{\mathbf{x^{\mathsf{T}} y}}{\| \mathbf{x} \| \hspace{1mm} \| \mathbf{y} \|}
$$

{\lit Hint: recall the angle between two vectors.}

\bigskip
You can think of the correlation between two variables as the cosine of the 
angle between two vectors.

\end{frame}

%------------------------------------------------

\end{document}
